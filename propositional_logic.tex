2. Propositional Logic
George Boole (1815-1864): "In 1854 he published An Investigation into the Laws of Thought, on Which are founded the
Mathematical Theories of Logic and Probabilities. Boole approached logic in a new way reducing it to a simple algebra,
incorporating logic into mathematics. He pointed out the analogy between algebraic symbols and those that represent logical
forms. It began the algebra of logic called Boolean algebra which now finds application in computer construction, switching
circuits etc." (according to MacTutor History of Mathematics archive).
See also:
G.Boole. The Calculus of Logic. The Cambridge and Dublin Mathematical Journal, vol. 3 (1848) (available online at
http://www.maths.tcd.ie/pub/HistMath/People/Boole/CalcLogic/, published by David R. Wilkins).
2.1. Proving Formulas Containing Implication only
Let us return to the Exercise 1.4.1(d), where you produced a sequence of 9 formulas proving the
following:
d) [L1, L2, MP]: A→(A→B) |- A→B.
Did you try the next step - proving of
d') [L1, L2, MP]: |- (A→(A→B))→(A→B)?
For proving directly – almost an impossible task!
Now, having deduction theorems, we can simplify the task of proving d), and make the task of proving d')
feasible. More precisely - the task of proving that d) and d') are provable. Indeed,
(1) A→(A→B) Hypothesis.
(2) A Hypothesis.
(3) A→B By MP, from (1), (2).
(4) B By MP, from (2), (3).
Thus, we have established that A→(A→B), A |- B. Now, by Deduction Theorem 1,
[L1, L2, MP]: A→(A→B) |- A→B.
And let us apply this theorem once more,
[L1, L2, MP]: |- (A→(A→B))→(A→B).
Note. In fact, we proved here only A→(A→B), A |- B, but we did not prove d) and d'), i.e. we did not
produce the corresponding sequences of formulas. We just proved that these sequences do exist! To
produce them really, we must apply the algorithm described in the proof of Deduction Theorem 1.
Exercise 2.1.1. Imagine applying the algorithm described in the proof of Deduction Theorem 1: a) to the
above 4 formula sequence – producing a sequence of 44 formulas proving [L1, L2, MP]: |-
(A→(A→B))→(A→B); b) to your 9 formula proof of (d) – producing a sequence of 29 formulas proving
the same.
Warning! Always be careful when selecting hypotheses. For example, to prove the strange
formula (the so-called Peirce's Law) |- ((A→B)→A)→A (it is provable in the classical logic, not in the
constructive logic!), you can try proving that (A→B)→A |- A, but not A→B, A |- A. Why? Because, by
Deduction Theorem 1, from A→B, A |- A it follows that A→B |- A→A and |- (A→B)→(A→A), or A |-
(A→B)→A and |- A→((A→B)→A). Where do you see |- ((A→B)→A)→A here?
Exercise 2.1.2. Prove the following [L1, L2, MP]:
a) |- ((A→B)→(A→C))→(A→(B→C)). What does this formula mean?
b) |- (A→B)→((B→C)→(A→C)). What does this formula mean? It's another version of the so-called
Law of Syllogism (by Aristotle), or the transitivity property of implication. Explain the difference
between this formula and Theorem 1.4.2: A→B, B→C |- A→C.
c) |- (A→(B→C))→(B→(A→C)). What does this formula mean? It's another version of the so-called
Premise Permutation Law. Explain the difference between this formula and Exercise 1.4.1(c):
A→(B→C) |- B→(A→C).
2.2. Proving Formulas Containing Conjunction
Theorem 2.2.1. a) [L5, MP] A, B |- A&B,
b) [L3, L4, MP]: A&B |- A, A&B |- B.
Let us prove (a).
(1) A Hypothesis.
(2) B Hypothesis.
(3) |- A→(B→A&B) Axiom L5: B→(C→B&C) with B = A, C =
                        B.
(4) B→A&B By MP, from (1) and (3).
(5) A&B By MP, from (2) and (4).
Now, let us prove (b).
(1) A&B Hypothesis.
(2) |- A&B→A Axiom L3: B&C→B with B = A, C = B.
(3) A By MP, from (1) and (2).
Thus, A&B |- A.
(1) A&B Hypothesis.
(2) |- A&B→B Axiom L4: B&C→C with B = A, C = B.
(3) B By MP, from (1) and (2).
Thus, A&B |- B.
Theorem 2.2.1 allows easy proving of equivalences. Let us recall that B↔C is defined as a shortcut for
(B→C)&(C→B). Of course, we will call B and C equivalent formulas, if and only if |- B↔C. For
example, by Theorem 1.4.1, [L1, L2, MP] |- A→A, hence, [L1, L2, L5, MP] |- (A→A)&(A→A), i.e.
[L1, L2, L5, MP] |- A↔A.
Of course, (a) of the Exercise 2.1.2 is the reverse formula of the axiom L2. Hence, by Theorem 2.2.1:
[L1, L2, L5, MP] |- (A→(B→C)) ↔ ((A→B)→(A→C)).
By (c) of the Exercise 2.1.2, and Theorem 2.2.1:
[L1, L2, L5, MP] |- (A→(B→C))↔(B→(A→C))
Now, let us prove another form of the Law of Syllogism, or Theorem 1.4.2 [L1, L2, MP]: A→B, B→C |-
A→C:
[L1-L4, MP] |- (A→B)&(B→C)→(A→C).
(1) (A→B)&(B→C) Hypothesis.
(2) |- (A→B)&(B→C)→(A→B) Axiom L3: B&C→B with B = A→B, C =
                                    B→C.
(3) |- (A→B)&(B→C)→(B→C) Axiom L4: B&C→C with B = A→B, C =
                                    B→C.
(4) A→B By MP, from (1) and (2).
(5) B→C By MP, from (1) and (3).
(6) A→C By by the transitivity property of implication
         (Theorem 1.4.2).
Thus, we have established that [L1-L4, MP]: (A→B)&(B→C) |- A→C. By Deduction Theorem 1, [L1-L4,
MP] |- ((A→B)&(B→C))→(A→C).
Exercise 2.2.1. Prove the following [L1- L5, MP]:
a) A→B, A→C |- A→B&C. What does it mean?
b) |- (A→B)&(A→C)→(A→B&C). What does it mean?
c) A→B&C |- A→B. What does it mean?
d) A→B&C |- A→C. What does it mean?
e) |- (A→B&C)→(A→B)&(A→C). What does it mean?
Hence,
[L1- L5, MP]: |- (A→B&C)↔(A→B)&(A→C).
Exercise 2.2.2. Prove the following, [L1- L5, MP]:
a) |- A&B↔B&A.What does it mean? That conjunction is commutative.
b) |- A&(B&C)↔(A&B)&C. What does it mean? That conjunction is associative.
c) |- A&A↔A. What does it mean? That conjunction is idempotent.
Exercise 2.2.3. Prove the following, [L1- L5, MP]:
a) |- (A→(B→C))↔(A&B→C). What does it mean?
b) |- (A→B)→(A&C→B&C). What does it mean? The converse formula (A&C→B&C)→(A→B) cannot
be true. Explain, why.
c) A |- B↔B&A. What does it mean?
Exercise 2.2.4. Let us recall that the equivalence connective A↔B is defined as a shortcut for
(A→B)&(B→A). Prove the following properties of this connective [L1- L5, MP]:
(a) |- A↔A (reflexivity),
(b) |- (A↔B)→(B↔A) (symmetricity),
(c) |- (A↔B)&(B↔C)→(A↔C) (transitivity).
2.3. Proving Formulas Containing Disjunction
Exercise 2.3.1. Prove the following [L1, L2, L6-L8, MP]:
a) [L8, MP]: A→C, B→C |- AvB→C. What does it mean?
b) [ L5, L6-L8, MP]: |- AvB↔BvA. What does it mean? That disjunction is commutative.
c) [L1, L2, L5, L6-L8, MP]: |- AvA↔A. What does it mean? That disjunction is idempotent.
Theorem 2.3.0. [L1, L2, L8, MP]: If there is a proof of A1, A2, ..., An, B |- D, and a proof of A1, A2, ...,
An, C |- D, then there is a proof of A1, A2, ..., An , BvC |- D.
Exercise 2.3.2. Prove Theorem 2.3.0.
Proving that disjunction is associative requires some sophistication (smart students: do not read the
proof below, try proving yourself):
[L1, L2, L5, L6-L8, MP]: |- Av(BvC)↔(AvB)vC.
First, imagine, we are trying to obtain, for example, Av(BvC)→(AvB)vC, from the axiom L8:
(A→(AvB)vC)→(((BvC)→(AvB)vC)→(Av(BvC)→(AvB)vC))).
It seems, the second premise (BvC)→(AvB)vC) also could be obtained from L8:
(B→(AvB)vC)→((C→(AvB)vC)→((BvC)→(AvB)vC))).
Thus, the → part of the task will be solved, if we could prove that |- A→Av(BvC), |- B→Av(BvC), |-
C→Av(BvC).
(1) |- A→Av(BvC)
Axiom L6: B→BvC with B = A, C = BvC.
Now, let us prove that |- B→Av(BvC).
(2) |- B→BvC Axiom L6.
(3) |- BvC→Av(BvC) Axiom L7: C→BvC with B = A, C = BvC.
(4) |- B→Av(BvC) From (2) and (3), by the transitivity property
                  of implication (Theorem 1.4.2).
Now, let us prove that |- C→Av(BvC).
(5) |- C→BvC Axiom L7.
(6) |- BvC→Av(BvC) Axiom L7: C→BvC with B = A, C = BvC.
(7) |- C→Av(BvC) From (5) and (6), by the transitivity property
                  of implication (Theorem 1.4.2).
Exercise 2.3.3. a) Prove the converse: [L1, L2, L6-L8, MP]: |- (AvB)vC)→Av(BvC).
b) Prove (use Deduction Theorem 1) that [L1, L2, L6-L8, MP]: |- (A→B)→(AvC→BvC). What does it
mean? The converse formula (AvC→BvC)→(A→B) cannot be true. Explain, why.
c) Prove that [L1, L2, L6-L8, MP]: |- A→B, C→D |- AvC→BvD. What does it mean?
The following theorem corresponds to the well-known distributive property of (number) addition to
multiplication: (a+b)c = ac+bc. Of course, the "dual" distributive property (i.e. - of multiplication to
addition) does not hold for numbers: ab+c=(a+c)(b+c) would imply ab+c=ab+ac+bc+cc, c=ac+bc+cc,
and, if c<>0, then 1=a+b+c. Still, surprisingly, in logic,
Theorem 2.3.1. Conjunction is distributive to disjunction, and disjunction is distributive to
conjunction:
[L1-L8, MP]: |- (A&B)vC↔(AvC)&(BvC).
[L1-L8, MP]: |- (AvB)&C↔(A&C)v(B&C).
First, let us prove that |- (A&B)vC→(AvC)&(BvC).
(1) Prove |- A&B→(AvC)&(BvC)
(2) Prove |- C→(AvC)&(BvC)
(3) |- (A&B)vC→(AvC)&(BvC)
From (1) and (2), by Exercise 2.3.1(a).
Exercise 2.3.4. a) Prove (1) and (2).
b) Do not read the following proof. Try proving yourself.
Now, let us prove the converse: |- (AvC)&(BvC)→(A&B)vC.
Note. The proof below starts with C as a hypothesis. Why not with (AvC)&(BvC)? Because, we will use
Deduction Theorem 1 to prove the intermediate formula (6) C→(BvC→(A&B)vC), not the final result!
(1) C Hypothesis.
(2) B→C From (1).
(3) |- C→(A&B)vC Axiom L7.
(4) B→(A&B)vC From (2) and (3).
(5) BvC→(A&B)vC From (4) and (3).
(6) |- C→(BvC→(A&B)vC) From (1)-(5), by Deduction Theorem 1.
(7) |- (B→A&B)→(BvC→(A&B)vC) Exercise 2.3.3(b).
(8) |- A→(B→A&B) Axiom L3.
(9) |- A→(BvC→(A&B)vC) From (8) and (7).
(10 From (9) and (6).
|- AvC→(BvC→(A&B)vC) 
    ) 
(11 From (10), by Exercise 2.2.3(a).
|- (AvC)&(BvC)→(A&B)vC 
  ) 
Now, we must prove that |- (AvB)&C→(A&C)v(B&C).
(1) Prove |- A→(C→(A&C)v(B&C))
(2) Prove |- B→(C→(A&C)v(B&C))
(3) Prove |- (AvB)&C→(A&C)v(B&C)
Exercise 2.3.5. Prove the above (1), (2) and (3).
Finally, we must prove that |- (A&C)v(B&C)→(AvB)&C.
Exercise 2.3.6. Prove that.
2.4. Formulas Containing Negation - Minimal Logic
Theorem 2.4.1. a) If [L1, L2, L9, MP]: A1, A2, ..., An, B |- C, and [L1, L2, L9, MP]: A1, A2, ..., An, B |-
¬C, then [L1, L2, L9, MP]: A1, A2, ..., An |- ¬B. What does it mean?
b) [L3, L4, L9, MP]: |- ¬(A&¬A). What does it mean? It's the so-called Law of Non-Contradiction.
Proof. a) By Deduction Theorem 1, A1, A2, ..., An |- B→C, and A1, A2, ..., An |- B→¬C. Let us continue
this proof by adding the axiom L9: (B→C)→((B→¬C)→¬B) as the next step. After this, by applying MP
twice we obtain ¬B. Q.E.D.
b) See Exercise 1.4.2 (e).
Exercise 2.4.1. a) (For smart students) Investigate the size (the number of formulas) of the proof [L1, L2,
L9, MP]: A1, A2, ..., An, |- ¬B as a function f(k, m) of the sizes k, m of the proofs [L1, L2, L9, MP]: A1,
A2, ..., An, B |- C and 1, A2, ..., An, B |- ¬C. You may wish to report your result. We will publish your
report on the web as an appendix to this book. The current record holder is Aiga Romane, 2008: f(k, m) ≤
3(k+m)+7. Improve this result, or prove that it is the best possible one.
b) [L1, L2, L9, MP]: A, ¬B |- ¬(A→B). Or, [L1-L4, L9, MP]: |- A&¬B → ¬(A→B). What does it mean?
c) |- [L1, L2, L9, MP]: (A→¬A)→¬A. What does it mean?
Attention: non-constructive reasoning! In Section 2.6, we will use the classical logic [L1-L11, MP] to
prove the converse formula of (c): ¬(A→B)→A&¬B, i.e. the equivalence ¬(A→B)↔A&¬B. This
formula cannot be proved in the constructive logic [L1-L10, MP] (see Section 2.8).
Theorem 2.4.2. [L1, L2, L9, MP]: |- (A→B)→(¬B→¬A). What does it mean? It's the so-called
Contraposition Law.
Note. The following form of Theorem 2.4.2 is called Modus Tollens:
[L1, L2, L9, MP]: |- A→B, ¬B |- ¬A.
Attention: non-constructive reasoning! In Section 2.6, we will use the classical logic [L1-L11, MP] to
prove the converse formula (¬B→¬A)→(A→B), i.e. the equivalence (A→B)↔(¬B→¬A). We will see
also that these formulas cannot be proved in the constructive logic [L1-L10, MP] (see Section 2.8).
Exercise 2.4.2. a) Prove Theorem 2.4.2.
b)Verify that, in our axiom system, the Law of Non-Contradiction and the Contraposition Law could be
used instead of the axiom L9. More precisely: prove L9 in the logic [L1-L5, Law of Non-Contradiction,
Contraposition Law, MP]. Be careful: do not use theorems depending on the axiom L9.
Theorem 2.4.3. [L1-L9, MP]: |- (A→¬B)↔(B→¬A). What does it mean?
First we prove that |- (A→¬B)→(B→¬A).
(1) A→¬B Hypothesis.
(2) B Hypothesis.
(3) (A→B)→((A→¬B)→¬A) Axiom L9: (B→C)→((B→¬C)→¬B) with B
                                         = A, C = B.
(4) A→B From (2) by Axiom L1 and MP.
(5) (A→¬B)→¬A From (3) and (4).
(6) ¬A
From (1) and (5).
Thus, by Deduction Theorem 1, |- (A→¬B)→(B→¬A). By swapping A and B we obtain the converse
formula: |- (B→¬A)→(A→¬B). Q.E.D.
Attention: non-constructive reasoning! Warning! The (very similar to Theorem 2.4.3) formula
(¬A→B)↔(¬B→A) can be proved only in the classical logic!
Theorem 2.4.4. [L1, L2, L9, MP]: |- A→¬¬A. What does it mean?
(1) A Hypothesis.
(2) (¬A→A)→((¬A→¬A)→¬¬A) Axiom L9.
(3) A→(¬A→A) Axiom L1.
(4) ¬A→A From (1) and (3) by MP.
(5) (¬A→¬A)→¬¬A From (2) and (4) by MP.
(6) ¬¬A From (5) and Theorem 1.4.1 by MP.
Attention: non-constructive reasoning! In Section 2.6, we will use the classical logic [L1-L11, MP] to
prove the converse formula |- ¬¬A→A, i.e. the equivalence |- ¬¬A↔A (the so-called Double Negation
Law). We will see also (Section 2.8) that these formulas cannot be proved in the constructive logic [L1-
L10, MP].
Still, in the minimal logic we can prove (Brouwer, 1923?):
Theorem 2.4.5. [L1, L2, L9, MP]: |- ¬¬¬A↔¬A. What does it mean?
Indeed, by Theorem 2.4.4, |- ¬A→¬¬¬A. By the Contraposition Law (Theorem 2.4.2), |-
(A→¬¬A)→(¬¬¬A→¬A). Hence, by Theorem 2.4.4, |- ¬¬¬A→¬A. Q.E.D.
Theorem 2.4.5 (and some of the following formulas in this and in the next section containing double
negations) may seem uninteresting to people believing unconditionally in the equivalence ¬¬A↔A. Still,
it seems interesting (at least - for a mathematician) to obtain a general characterization of logical formulas
that do not depend on the Law of the Excluded Middle. In Section 2.7 we will use these formulas to prove
the elegant and non-trivial Glivenko's theorem: a) A is provable in the classical propositional logic (i.e. in
[L1-L11, MP]), if and only if ¬¬A is provable in the constructive propositional logic (i.e. in [L1-L10,
MP]), b) ¬A is provable in the classical propositional logic, if and only if ¬A is provable in the
constructive propositional logic.
Theorem 2.4.6. a) [L1, L2, L9, MP]: |- (¬A→A)→¬¬A. What does it mean?
b) [L1, L2, L6, L7, L9, MP]: |- ¬¬(Av¬A). What does it mean? In this weak form, the Law of the Excluded
Middle can be "proved constructively". The axiom L11 can't be used in this proof! Explain, why. (Hint:
derive a contradiction from ¬(Av¬A).)
Exercise 2.4.3. Prove (a) and (b) of Theorem 2.4.6.
Theorem 2.4.7. [L1-L9, MP]: a) |- (A→B)→(¬¬A→¬¬B). What does it mean?
b) |- ¬¬(A→B)→(¬¬A→¬¬B). What does it mean?
c) |- (A→(B→C))→(¬¬A→(¬¬B→¬¬C)). What does it mean?
d) ¬¬(A→B), ¬¬(B→C) |- ¬¬(A→C). What does it mean?
e) ¬¬A, ¬¬(A→B) |- ¬¬B. What does it mean?
The converse of (a): (¬¬A→¬¬B)→(A→B) cannot be proved in the constructive logic (see Section 2.8).
To prove (a), we must simply apply twice the Contraposition Law: (A→B)→(¬B→¬A)→(¬¬A→¬¬B).
And, of course, (e) is an easy consequence of (b).
Now, let us prove (b).
(1) ¬¬(A→B) Hypothesis.
(2) ¬¬A Hypothesis.
(3) |- ¬¬A→((A→B)→¬¬B) From (a), by transposing A→B and
                                  ¬¬A, by the Premise Permutation Law.
(4) (A→B)→¬¬B From (2) and (3).
(5) |- ((A→B)→¬¬B)→(¬¬¬B→¬(A→B)) By the Contraposition Law.
(6) ¬¬¬B→¬(A→B) From (4) and (5).
(7) |- (¬¬¬B→¬(A→B))→(¬¬(A→B)→¬¬¬¬B) By the Contraposition Law.
(8) ¬¬(A→B)→¬¬¬¬B From (6) and (7).
(9) ¬¬¬¬B From (1) and (8).
(10 By Theorem 2.4.5.
¬¬¬¬B→¬¬B 
        ) 
(11 From (9) and (10).
¬¬B 
  ) 
Thus, by Deduction Theorem 1, |- ¬¬(A→B)→(¬¬A→¬¬B).
Let us prove (c).
(1) A→(B→C) Hypothesis.
(2) ¬¬A Hypothesis.
(3) ¬¬B Hypothesis.
(4) ¬¬A→¬¬(B→C) From (1), by (a).
(5) ¬¬(B→C) From (2) and (4).
(6) ¬¬B→¬¬C From (5), by (b).
(7) ¬¬C From (3) and (6).
Thus, by Deduction Theorem 1, |- (A→(B→C))→(¬¬A→(¬¬B→¬¬C)).
Now we can prove (d). First, let us take (c) with A = A→B, B = B→C, C = A→C:
(1) |- ((A→B)→((B→C)→(A→C)))→(¬¬(A→B)→(¬¬(B→C)→¬¬(A→C))).
(2) |- (A→B)→((B→C)→(A→C) By transitivity of implication and Deduction
                                   Theorem 1.
(3) ¬¬(A→B) Hypothesis.
(4) ¬¬(B→C) Hypothesis.
(5) ¬¬(A→C) From (1), (3) and (4).
Theorem 2.4.8. [L1-L9, MP]: a) |- ¬¬(A&B)↔(¬¬A & ¬¬B). What does it mean?
b) |- ¬¬A v ¬¬B → ¬¬(AvB). What does it mean?
Attention: non-constructive reasoning! The converse of (b): ¬¬(AvB) → ¬¬A v ¬¬B cannot be proved
in the constructive logic (see Section 2.8). What does it mean? If we simply succeed in deriving a
contradiction from ¬(AvB), then, perhaps, we do not have a method allowing to decide, which part of
¬¬A v ¬¬B is true - ¬¬A, or ¬¬B?
Exercise 2.4.4. Prove Theorem 2.4.8. (Hint: use the result of Exercise 2.2.3(a), if needed.)
Theorem 2.4.9. [L1, L2, L9, MP] |- ¬A→(A→¬B) (compare with Exercise 1.4.2(d)). What does it mean?
It's a weak form of the "crazy" axiom L10: ¬A→(A→B). This axiom says: "Contradiction implies
anything". In the minimal logic we can prove 50% of L10: "Contradiction implies that all is wrong". Of
course, this 50%-provability of L10 decreases the significance of the minimal logic accordingly.
Proof. See Exercise 2.4.5.
Theorem 2.4.10. [L1-L9, MP]:
a) |- ¬Av¬B → ¬(A&B). It's a half of the so-called First de Morgan Law. What does it mean?
b) |- ¬(AvB) ↔ ¬A&¬B. It's the so-called Second de Morgan Law. What does it mean?
Attention: non-constructive reasoning! The second half of (a) – the converse implication, i.e. the
equivalence ¬(A&B) ↔ ¬Av¬B can be proved in the classical logic, yet not in the constructive logic (see
Section 2.8). Explain, why.
Augustus de Morgan (1806-1871): "He recognised the purely symbolic nature of algebra and he was aware of the existence of
algebras other than ordinary algebra. He introduced de Morgan's laws and his greatest contribution is as a reformer of
mathematical logic." (according to MacTutor History of Mathematics archive).
Use Contraposition Law to prove (a) and (b→) in Exercise 2.4.5.
Let us prove (b←).
(0) ¬A&¬B Hypothesis.
(1) ¬A From (0), by Axiom L3.
(2) ¬B From (0), by Axiom L4.
(3) A→¬C From (1), by Theorem 2.4.9: ¬A→(A→¬C).
                 C is any formula.
(4) B→¬C From (2), by Theorem 2.4.9: ¬B→(B→¬C).
                 C is any formula.
(5) AvB→¬C From (3) and (4), by Axiom L8:
             (A→¬C)→((B→¬C)→(AvB→¬C).
(6) AvB→¬¬C Repeat (3)-(5) with ¬¬C instead of ¬C.
(7) ¬(AvB) From (5) and (6), by Axiom L9:
           (AvB→¬C)→((AvB→¬¬C)→¬(AvB)).
Thus, by [L1, L2] Deduction Theorem 1, [L1-L9, MP] |- ¬A&¬B → ¬(AvB).
Exercise 2.4.5. Prove:
a) Theorem 2.4.9.
b) (a) and (b→) of Theorem 2.4.10. (Hint: use Contraposition Law).
c) [L1-L9, MP]: |- (A→B)→¬(A&¬B). What does it mean? Compare with Exercise 2.4.1.
d) [L1-L8, MP]: |- AvB → ((A→B)→B). What does it mean?
Attention: non-constructive reasoning! The converse implication of (a), ¬(A&¬B)→(A→B) cannot be
proved in the constructive logic (see Section 2.8). Explain, why. Still, we will prove this formula in the
classical logic.
The converse of (b): ((A→B)→B) → AvB cannot be proved in the constructive logic (see Section 2.8).
Explain, why. Still, we will prove this formula in the classical logic.
Thus, the formula ¬¬(Bv¬B) can be proved in the constructive logic, but Bv¬B can't - as we will see in
Section 2.8.
2.5. Formulas Containing Negation - Constructive Logic
In this book, constructive logic is used as a synonym of intuitionistic logic!
Constructive logic includes the "crazy" axiom L10: ¬B→(B→C), but rejects the Law of the Excluded
Middle L11: Bv¬B as a general logical principle.
Theorem 2.5.1. a) [L10, MP]: A, ¬A |- B. What does it mean?
b) [L1, L2, L8, L10, MP]: |- AvB→(¬A→B). What does it mean?
c) [L1, L8, L10, MP]: |- ¬AvB→(A→B). What does it mean?
Of course, (a) follows directly from L10, by MP.
Exercise 2.5.1. Prove (b) and (c) of Theorem 2.5.1. Note: when proving (c), you cannot use Deduction
Theorem 1 (because of the missing axiom L2). So, simply build a sequence of 5 formulas representing the
proof of (c).
Attention: non-constructive reasoning! The converse of (b), i.e. (¬A→B)→AvB cannot be proved in
the constructive logic (see Section 2.8). Explain, why. The converse of (c), i.e. (A→B)→¬AvB cannot be
proved in constructive logic (see Section 2.8). Explain, why.
Surprisingly, (b), i.e. the rule AvB, ¬A |- B seems to be a quite a "natural" logical principle, yet it cannot
be proved without the "crazy" axiom L10! Why not? Because it implies L10! Indeed,
1 AvB→(¬A→B) Hypothesis.
2 ¬A Hypothesis.
3 A Hypothesis.
4 A→AvB Axiom L6.
5 AvB By MP, from (3) and (4).
6 B By MP, from (1), (5) and (2).
Hence, by Deduction Theorem 1, [L1, L2, L6, MP]: AvB→(¬A→B) |- ¬A→(A→B).
In Section 2.8 we will prove that L10 cannot be derived from L1-L9, hence, (b) also cannot be derived
from L1-L9 (i.e. without L10).
Theorem 2.5.2. [L1-L10, MP]: a) |- (¬¬A→¬¬B)→¬¬(A→B). It's the converse of Theorem 2.4.7(b).
Hence, [L1-L10, MP]: |- ¬¬(A→B)↔(¬¬A→¬¬B).
b) |- ¬¬A→(¬A→A). It's the converse of Theorem 2.4.6(a). Hence, [L1-L10, MP]: |- ¬¬A↔(¬A→A).
c) |- Av¬A→(¬¬A→A). What does it mean?
d) |- ¬¬(¬¬A→A). What does it mean?
Of course, (b) is an instance of the axiom L10.
To prove (a), let us prove that [L1-L10, MP]: ¬¬A→¬¬B, ¬(A→B) |- ¬B, ¬¬B. Then, by Theorem 2.4.1,
(a) |- (¬¬A→¬¬B)→¬¬(A→B).
Exercise 2.5.2. a) Prove that [L1-L10, MP]: ¬¬A→¬¬B, ¬(A→B) |- ¬B, ¬¬B.
b) Prove (c) and (d) of Theorem 2.5.2.
Exercise 2.5.3. Prove that in [L1-L10, MP]:
a) A |- B↔Bv¬A. What does it mean?
b) |- Bv(A&¬A) ↔ B. What does it mean?
c) |- ((A&¬A)&B)vC ↔ C. What does it mean?
2.6. Formulas Containing Negation - Classical Logic
If you agree to adopt the formula Bv¬B, i.e. the Law of the Excluded Middle (Axiom L11 in the list of
Section 1.3), you can prove, first of all, the so called Double Negation Law:
Theorem 2.6.1. [L1-L11, MP]: |- ¬¬A → A. Hence, [L1-L11, MP]: |- ¬¬A ↔ A.
Indeed, by Theorem 2.5.2, [L1-L10, MP]: |- Av¬A→(¬¬A→A), hence, [L1-L11, MP]: |- ¬¬A→A. Q.E.D.
In the minimal logic we proved Theorem 2.4.4: [L1, L2, L9, MP]: |- A→¬¬A. Hence, [L1-L11, MP]: |-
¬¬A ↔ A.
Attention: non-constructive reasoning! The formula ¬¬A→A cannot be proved in the constructive
logic, see Section 2.8. Why? Because it represents a kind of non-constructive reasoning. Indeed, imagine,
you wish to prove that ∃xB(x). Assume the contrary, ¬∃xB(x), and derive a contradiction. Thus you have
proved... the negation of ¬∃xB(x), i.e. ¬ ¬∃xB(x). To conclude ∃xB(x) from ¬ ¬∃xB(x), you need the
Double Negation Law. Hence, by adopting this law as a logical principle, you would allow non-
constructive existence proofs – if you prove ∃xB(x) by assuming ¬∃xB(x), and deriving a contradiction,
then you may not obtain a method allowing to find a particular x satisfying B(x).
Exercise 2.6.1. Prove that [L8, L11, MP]: A→B, ¬A→B |- B. Or, by Deduction Theorem 1, [L1, L2, L8,
L11, MP]: (A→B)→((¬A→B)→B). What does it mean? This formula cannot be proved in the
constructive logic (see Section 2.8). Explain, why.
In the classical logic, you can prove also the full form of the Contraposition Law:
Theorem 2.6.2. [L1-L11, MP]: |- (A→B) ↔ (¬B→¬A).
We proved a half of this Law in the minimal logic as Theorem 2.4.2: [L1, L2, L9, MP]: |-
(A→B)→(¬B→¬A). Let us prove the remaining half: [L1-L11, MP] |- (¬B→¬A) → (A→B).
(1) ¬B→¬A Hypothesis.
(2) A Hypothesis.
(3) ¬¬A→¬¬B From (1), by the first half.
(4) |- A→¬¬A Double Negation Law.
(5) |- ¬¬B→B Double Negation Law.
(6) B From (4), (3) ans (5).
By Deduction Theorem 1, [L1-L11, MP] |- (¬B→¬A) → (A→B).
Attention: non-constructive reasoning! The formula (¬B→¬A) → (A→B) cannot be proved in the
constructive logic, see Section 2.8. Explain, why.
Exercise 2.6.1A. Prove that in [L1-L11, MP]:
a) |- (¬A→B)↔(¬B→A) (compare with Theorem 2.4.3).
b) |- (A→B)→((¬A→¬B)→(B↔A)).
Attention: non-constructive reasoning! These two formulas cannot be proved in the constructive logic,
see Section 2.8.
Theorem 2.6.3. [L1-L11, MP]: |- ¬(A&B) ↔ ¬Av¬B. It's the so-called First de Morgan Law.
A half of this Law we proved in the minimal logic as Theorem 2.4.10(a): [L1-L9, MP] |- ¬Av¬B →
¬(A&B). Let us prove the remaining half: [L1-L11, MP] |- ¬(A&B) → ¬Av¬B.
Attention: non-constructive reasoning! This formula cannot be proved in the constructive logic, see
Section 2.8. Explain, why.
Let us start by proving ¬( ¬Av¬B) → A&B.
(1) ¬( ¬Av¬B) Hypothesis.
(2) |- ¬( ¬Av¬B)→¬¬A&¬¬B By the Second de Morgan Law -Theorem
                                 2.4.10(b).
(3) |- ¬¬A&¬¬B → ¬¬(A&B) Theorem 2.4.8(a). [L1-L9, MP]!
(4) ¬¬(A&B) From (1), (2) and (3).
Thus, by Deduction Theorem 1, [L1-L9, MP] |- ¬( ¬Av¬B) → ¬¬(A&B). By applying the first half of the
Contraposition Law (provable in the minimal logic): [L1-L9, MP] |- ¬¬¬(A&B) →¬ ¬(¬Av¬B). By
Theorem 2.4.5: [L1-L9, MP] |- ¬(A&B)→¬¬¬(A&B), hence, [L1-L9, MP] |- ¬(A&B)→¬ ¬(¬Av¬B).
Now, by the Double Negation Law, [L1-L11, MP] |- ¬¬(¬Av¬B)→¬Av¬B, hence, [L1-L11, MP] |-
¬(A&B)→¬Av¬B. Q.E.D.
In the classical logic, we can express implication by negation and disjunction. Indeed, we already know
that [L1-L10, MP]: |- ¬AvB→(A→B) (Theorem 2.5.1(c)).
Theorem 2.6.4. a) [L1-L8, MP]: AvC |- (A→B)→BvC. Hence, [L1-L8, MP]: Av¬A |- (A→B)→¬AvB.
b) [L1-L11, MP]: |- (A→B)↔¬AvB.
Of course, (b) follows from (a) and Theorem 2.5.1(c). Let us prove (a).
(1) A, A→B |- B
(2) A, A→B |- BvC By Axiom L6.
(3) A |- (A→B)→BvC By Deduction Theorem 1.
(4) C, A→B |- C
(5) C, A→B |- BvC By Axiom L7.
(6) C |- (A→B)→BvC By Deduction Theorem 1.
(7) AvC |- (A→B)→BvC
By Axiom L8.
Exercise 2.6.2. Prove that in [L1-L11, MP]:
a) |- B&(Av¬A) ↔ B. What does it mean?
b) |-( (Av¬A)vB)&C ↔ C. What does it mean?
c) |- ((A→B)→B) → AvB. What does it mean? Hence, by Exercise 2.4.5(d), [L1-L11, MP]: |-
((A→B)→B) ↔ AvB.
Exercise 2.6.3. Prove that in [L1-L11, MP]:
a) |- (A→B) ↔ ¬(A&¬B). What does it mean?
b) |- ¬(A→B) ↔ A&¬B. What does it mean?
c) |- AvB ↔ (¬A→B). What does it mean?
d) |- A&B ↔ ¬(A→¬B). What does it mean?
e) (For smart students) Try detecting, which parts of these equivalences are provable: 1) in the minimal
logic, 2) in the constructive logic.
Strange formulas
Exercise 2.6.4. Prove in [L1-L11, MP] the following strange formulas:
a) |- Av(A→B). What does it mean? Does it mean anything at all?
b) |- (A→B)v(B→A). What does it mean? Does it mean anything at all? The most crazy theorem of the
classical propositional logic?
c) |- ((A→B)→A)→A. What does it mean? Does it mean anything at all? It is the so-called Peirce's Law
from:
C. S. Peirce. On the algebra of logic: A contribution to the philosophy of notation. American Journal of Mathematics, 1885,
vol.7, pp.180-202.
2.7. Constructive Embedding. Glivenko's Theorem
Let us recall some of the results of previous sections concerning double negations:
Theorem 2.4.4. [L1, L2, L9, MP]: |- A→¬¬A.
Theorem 2.4.5. [L1-L9, MP]: |- ¬¬¬A↔¬A.
Theorem 2.4.6(b). [L1-L9, MP]: |- ¬¬(Av¬A). In this weak form, the Law of the Excluded Middle can be
"proved constructively".
Theorem 2.4.7. [L1-L9, MP]: a) |- (A→B)→(¬¬A→¬¬B).
b) |- ¬¬(A→B)→(¬¬A→¬¬B).
c) |- (A→(B→C))→(¬¬A→(¬¬B→¬¬C)).
d) ¬¬(A→B), ¬¬(B→C) |- ¬¬(A→C).
e) ¬¬A, ¬¬(A→B) |- ¬¬B.
Theorem 2.4.8. [L1-L9, MP]: a) |- ¬¬(A&B)↔(¬¬A & ¬¬B).
b) |- ¬¬A v ¬¬B → ¬¬(AvB).
Theorem 2.5.2. [L1-L10, MP]: a) |- (¬¬A→¬¬B)→¬¬(A→B). It's the converse of Theorem 2.4.7(b).
d) |- ¬¬(¬¬A→A).
Theorem 2.6.1. [L1-L11, MP]: |- ¬¬A ↔ A.
Does it mean that for any formula A: if [L1-L11, MP]: |- A, then [L1-L10, MP]: |- ¬¬A? (The converse is
obvious: if [L1-L10, MP]: |- ¬¬A, then [L1-L11, MP]: |- A by Theorem 2.6.1.)
Imagine, we have a proof of [L1-L11, MP]: |- A. It is a sequence of formulas R1, R2, ..., Rn, where Rn = A.
If this sequence does not contain instances of the axiom L11, then it is a proof of [L1-L10, MP]: |- A as
well. Hence, according to Theorem 2.4.4, [L1-L10, MP]: |- ¬¬A
If the sequence R1, R2, ..., Rn contains some instances of L11, i.e. formulas having the form Bv¬B, then,
according to Theorem 2.4.6(b), we could try replacing each such formula by a sequence proving that [L1-
L9, MP]: |- ¬¬(Bv¬B). It appears that each of the formulas ¬¬R1, ¬¬R2, ..., ¬¬Rn is provable in [L1-L10,
MP].
a) If Rk is an instance of the axioms L1-L10, then [L1-L10, MP]: |- ¬¬Rk (Theorem 2.4.4).
b) If Rk is an instance of the axiom L11, then [L1-L10, MP]: |- ¬¬Rk (Theorem 2.4.6(b)).
c) Now, let us assume that i, j < k, and Ri, Rj |- Rk directly by MP, i.e. Rj is Ri→Rk. We know already that
[L1-L10, MP]: |- ¬¬Ri and [L1-L10, MP]: |- ¬¬(Ri→Rk). By Theorem 2.4.7(b), [L1-L9, MP]: |-
¬¬(Ri→Rk)→(¬¬Ri→¬¬Rk). Hence, [L1-L10, MP]: |- ¬¬Rk.
Because A = Rn, we have proved the remarkable Glivenko's theorem from 1929:
V.Glivenko. Sur quelques points de la logique de M. Brouwer. Academie Royale de Belgique, Bulletins de la classe des
sciences, 1929, ser.5, vol.15, pp.183-188.
Valery Ivanovich Glivenko (1896-1940, see http://www.math.ru/history/people/glivenko, in Russian) is best known by the so-
called Glivenko-Cantelli theorem in probability theory.
Glivenko's Theorem. [L1-L11, MP]: |- A, if and only if [L1-L10, MP]: |- ¬¬A. Or: a formula A is provable
in the classical propositional logic, if and only if its double negation ¬¬A is provable in the constructive
propositional logic.
This theorem provides a kind of a "constructive embedding" for the classical propositional logic: any
classically provable formula can be "proved" in the constructive logic, if you simply put two negations
before it.
Corollary. [L1-L11, MP]: |- ¬A, if and only if [L1-L10, MP]: |- ¬A. Or: a "negative" formula ¬A is
provable in the classical propositional logic, if and only if it is provable in the constructive propositional
logic.
Indeed, if [L1-L11, MP]: |- ¬A, then by Glivenko's theorem, [L1-L10, MP]: |- ¬¬¬A, and by Theorem
2.4.5, [L1-L10, MP]: |- ¬A. Q.E.D.
Exercise 2.7.1. Prove the following version of Glivenko's theorem (see Kleene [1952]):
a) If [L1-L11, MP]: A1, A2, ..., An |- C, then [L1-L10, MP]: ¬¬A1, ¬¬A2, ..., ¬¬An |- ¬¬C.
b) If [L1-L11, MP]: ¬A1, ¬A2, ..., ¬An, B1, B2, ..., Bp |- ¬C, then [L1-L10, MP]: ¬A1, ¬A2, ..., ¬An , ¬¬B1,
¬¬B2, ..., ¬¬Bp |- ¬C.
2.8. Axiom Independence. Using Computers in Mathematical Proofs
If one of our axioms Li could be proved by using the remaining n-1 axioms, then we could simplify our
logical system by dropping Li as an axiom. A striking example:
Theorem 2.8.1. The axiom L9: (A→B)→((A→¬B)→¬A) can be proved in [L1, L2, L8, L10, L11, MP].
This fact was established by Augusts Kurmitis (on the web, also: A. A. Kurmit):
A. A. Kurmitis. On independence of a certain axiom system of the propositional calculus. Proc. Latvian State University,
1958, Vol. 20, N3, pp. 21-25 (in Russian).
The following proof of L9 in [L1, L2, L8, L10, L11, MP] is due to Janis Sedols (1939-2011).
First, let us establish that the formula (A→¬A)→¬A can be proved in [L1, L2, L8, L10, L11, MP] (in
Exercise 2.4.1 we established that [L1, L2, L9, MP]: |- (A→¬A)→¬A):
(1)
|-
Axiom L8.
(A→¬A)→((¬A→¬A)→(Av¬A)→¬A))
(2) A→¬A Hypothesis.
(3) |- ¬A→¬A This is provable in [L1, L2, MP] (Theorem
                1.4.1).
(4) Av¬A Axiom L11.
(4) ¬A From (1), (2), (3) and (4), by MP.
(6) |- (A→¬A)→¬A By Deduction Theorem 1 (which is valid for
                      any logical system containing [L1, L2, MP]).
Now let us establish that in [L1, L2, L10, MP]: A→B, A→¬B |- A→¬A.
(7) A→B Hypothesis.
(8) A→¬B Hypothesis.
(9) A Hypothesis.
(9) B From (7), (9), by MP.
(10 From (8), (9), by MP.
¬B 
 ) 
(11 Axiom L10.
|- ¬B→(B→¬A) 
      ) 
(12 From (9), (10) and (11) by MP.
¬A 
 ) 
(13 By Deduction Theorem 1 (which is valid for
A→B, A→¬B |- A→¬A any propositional system containing [L1, L2,
        ) MP]).
Finally, let us merge the proofs of (6) and (13), then by MP we obtain ¬A, i.e.
[L1, L2, L8, L10, L11, MP]: A→B, A→¬B |- ¬A.
Now, by Deduction Theorem 1 (which is valid for any propositional system containing [L1, L2, MP]) we
obtain the axiom L9:
[L1, L2, L8, L10, L11, MP]: (A→B)→((A→¬B)→¬A).
Q.E.D.
What should we do after establishing that one of our axioms is "dependent"?
Do you think, we should drop L9 as an axiom of our logical system?
First, let's note that we have proved L9 by using three problematic axioms: L1, L10, L11. But L9 itself is
not problematic!
Secondly, L9 cannot be proved in [L1-L8, L10, MP] (see Theorem 2.8.2 below). Hence, if we would drop
L9, then, instead of a simple definition
"classical logic = constructive logic + L11",
we would have a more complicated one:
"constructive logic = classical logic – L11 + L9".
Now, the question of questions:
Is Law of Excluded Middle an independent logical principle?
I.e., could we prove the Law of the Excluded Middle (the axiom L11: Bv¬B) by using the other axioms
(i.e. [L1-L10, MP]) as we proved L9 in [L1, L2, L8, L10, L11, MP]? If not, how could we demonstrate that
this is impossible at all? How could we demonstrate that some logical principle is independent, i.e. that it
cannot be derived from other principles?
Let us assume, we have an algorithm q computing for each formula A some its "property" q(A) such that:
a) q(L1) is true, q(L2) is true, ..., q(L10) is true (i.e. the axioms L1-L10 possess property q).
b) If q(A) is true and q(A→B) is true, then q(B) is true (i.e. MP "preserves" property q). Hence, q(F) is
true for all the formulas F that are provable in [L1-L10, MP].
c) q(L11) is false (L11 does not possess property q).
If we could obtain such a property q, then, of course, this would demonstrate that L11 cannot be proved in
[L1-L10, MP], i.e. that the Law of the Excluded Middle is an independent logical principle.
The most popular way of introducing such properties of formulas are the so-called "multi-valued logics"
or "many-valued logics", introduced by Jan Lukasiewicz and Emil Post:
J.Lukasiewicz. O logice trojwartosciowej. Ruch Filozoficzny (Lwow), 1920, vol. 5, pp. 169-171
E.Post. Introduction to a general theory of elementary propositions. Amer. journ. math., 1921, vol. 21, pp.163-195
Read more: Many-Valued Logic by Siegfried Gottwald in Stanford Encyclopedia of Philosophy.
For example, let us consider a kind of "three-valued logic", where 0 means "false", 1 - "unknown" (or
NULL - in terms of SQL), 2 - "true". Then it would be natural to define conjunction and disjunction as
A&B = min(A,B)
AvB = max(A,B).
But how should we define implication and negation?
A B A&B AvB A→B
0 0 0 0 i1
0 1 0 1 i2
0 2 0 2 i3
1 0 0 1 i4
1 1 1 1 i5
1 2 1 2 i6
2 0 0 2 i7
2 1 1 2 i8
2 2 2 2 i9
A ¬A
0 i10
1 i11
2 i12
Thus, theoretically, we have here to explore: 39 = 19683 variants of implication definitions and 33 = 27
negation definitions.
Do you think, it would be natural to set the values of ¬A as follows?
A ¬A
0 2
1 1
2 0
Yes, if we would try building a "natural" three-valued logic, in which "1" would mean, indeed,
"unknown". To fill in the “natural” table of three-valued implication, we could use, for example, the
classical equivalence (A→B)↔¬AvB. In this way we could obtain the “natural” three-valued logic used,
for example, for handling of NULL-values in SQL.
However, our aim is here, in a sense, just the opposite of “natural”. We will consider
"under the above truth tables, formula A always takes "true" values"
as a kind of the above-mentioned "property" q(A). Hence, we will try to define the tables for implication
and negation in such a way that:
a) the axioms L1, L2, ..., L10 always take "true" values (i.e. 2),
b) Modus Ponens "preserves" taking always "true" values (i.e. if the formulas A and A→B are always 2,
then B also is always 2),
c) the axiom L11 sometimes takes the values 0 or 1.
Because of "violating" L11, the definitions of implication and negation, having these properties, cannot be
100% natural. So, we must explore (at least some of) the "unnatural" versions as well.
Exercise 2.8.1. Develop a simple (recursive) computer program receiving as input:
a) Any such "truth tables".
b) Any formula F consisting of letters A, B, C, and propositional connectives.
and printing out "truth values" of the formula F, for example, if F = B→(A→B):
A B B→(A→B)
0 0 2
0 1 2
0 2 2
1 0 2
1 1 2
1 2 2
2 0 2
2 1 2
2 2 2
In this example the axiom L1 always takes "true" values. Perhaps, we should be interested only in those
variants of our "truth tables" that "satisfy" at least the axioms L1, L2, ..., L8 forcing them always to take
"true" values.
Note. See my version of the program in C++: header file, implementation, download the entire Borland C++ project (about
200K zip).
Thus, we consider
"under the above truth tables, formula A always takes "true" values"
as a kind of the "property" q(A).
Will Modus Ponens preserve this property? If A is "true", and A→B is "true", how could B be not? Let us
consider the relevant part of the above truth tables (i.e. the part where A is "true"):
A B A→B
2 0 i7
2 1 i8
2 2 i9
If we would consider only those variants of our "truth tables" where i7 = 0 or 1, i8 = 0 or 1, and i9 = 2,
then, if B would not be 2 for some values of its arguments, then A→B also would not be 2 for the same
values of arguments.
Hence, if we restrict ourselves to "truth tables" with i7 = 0 or 1, i8 = 0 or 1, and i9 = 2, then MP preserves
the property of "being true". I.e., from "true" formulas MP can derive only "true" formulas.
The next idea: if we wish the axiom L6: A→AvB always taking the value "true" (i.e. the value 2), then, if
A≤B, then A→B must be 2.
Thus, of all the 39 = 19683 possible implication definition variants only the following 3*2*2 = 12 variants
are worth of exploring:
A B A→B
0 0 2
0 1 2
0 2 2
1 0 i4=0,1,2
1 1 2
1 2 2
2 0 i7=0,1
2 1 i8=0,1
2 2 2
Exercise 2.8.2. a) Verify that under any of these 12 implication definitions the axioms L3, L4, L6, L7
always take the value 2, i.e. you do not need testing these axioms any more.
b) For each of the axioms, L1, L2, L5 and L8, determine all the possible combinations of the values of i4,
i7, i8 forcing it to take always the value 2.
Note. The "intersection" of b) consists of 5 combinations (see the results file #00).
Exercise 2.8.3. Extend your previous computer program by adding 6 nested loops: for i4=0 to 2, for i7=0
to 1, for i8=0 to1, for iaa=0 to 2, for ib=0 to 2, for ic=0 to 2. Let the program print out only those variants
of "truth tables" that make "true" all the axioms L1-L8. (My program yields 135 such variants, see the
results file #00).
Thus, now we have 135 variants of "truth tables" that make "true" all the axioms L1-L8. Let us search
among them for the variants that allow proving of axiom independency results we are interested in.
Axiom L9
In Theorem 2.8.1 we established that the axiom L9: (A→B)→((A→¬B)→¬A) can be proved in [L1-L8,
L10, L11, MP]. Still,
Theorem 2.8.2. The axiom L9 cannot be proved in [L1-L8, L10, MP].
Proof. Let your program print out only those variants of "truth tables" that make "true" all the axioms L1-
L8, and make: L9 - not "true", and L10 - "true". My program yields 66 such variants, see the results file
#01. I like especially the (most natural?) variant #33:
Implication variant #3:
2 2 2 2 2 2 0 1 2 L1-L8 true.
Variant #33. Negation: 2 1 0 L9 not true. L10 true. L11 not true.
A B A→B
0 0 2
0 1 2
0 2 2
1 0 2
1 1 2
1 2 2
2 0 0
2 1 1
2 2 2
A ¬A
0 2
1 1
2 0
See the extended results file #1 for this variant.
Under this variant the axioms L1-L8 and L10 are "true". As we know, under this variant, by MP, from
"true" formulas only "true" formulas can be derived. The axiom L9 is not "true" under this variant:
A B (A→B)→((A→¬B)→¬A)
0 0 2
0 1 2
0 2 2
1 0 1
1 1 1
1 2 1
2 0 2
2 1 2
2 2 2
Hence, L9 cannot be proved in [L1-L8, L10, MP]. Q.E.D.
In a similar way, we can obtain other independence results.
Axiom L10
Theorem 2.8.3. The "crazy" axiom L10: ¬B→(B→C) cannot be proved in the minimal logic [L1-L9,
MP], and even not in [L1-L9, L11, MP].
Proof. Let your program print out only those variants of "truth tables" that make "true" all the axioms L1-
L8, and make: L9 - "true", L10 - not "true", and L11 - "true". My program yields 6 such variants, see the
results file #02. I like especially the (somewhat natural?) variant #1:
Implication variant #1:
2 2 2 0 2 2 0 1 2 L1-L8 true.
Variant #1. Negation: 2 2 1 L9 true. L10 not true. L11 true.
See the extended results file #2 for this variant.
Under this variant the axioms L1-L9 and L11 are "true". As we know, under this variant, by MP, from
"true" formulas only "true" formulas can be derived. The axiom L10 is not "true" under this variant:
A B ¬A→(A→B)
0 0 2
0 1 2
0 2 2
1 0 2
1 1 2
1 2 2
2 0 0
2 1 1
2 2 2
Hence, L10 cannot be proved in [L1-L9, L11, MP]. Q.E.D.
Axiom L11
Now, let us prove the main result of this section:
Theorem 2.8.4. The Law of the Excluded Middle L11: Bv¬B cannot be proved in the constructive
propositional logic [L1-L10, MP]. I.e. the Law of the Excluded Middle is an independent logical
principle.
Proof. Let your program print out only those variants of "truth tables" that make "true" all the axioms L1-
L8, and make: L9 - "true", L10 - "true", L11 - not "true". My program yields only one such variant, see the
results file #03:
Implication variant #1:
2 2 2 0 2 2 0 1 2 L1-L8 true.
Variant #1. Negation: 2 0 0 L9 true. L10 true. L11 not true.
See the extended results file #3 for this variant.
Under this variant the axioms L1-L10 are "true". As we know, under this variant, by MP, from "true"
formulas only "true" formulas can be derived. The axiom L11 is not "true" under this variant:
B ¬B Bv¬B
0 2 2
1 0 1
2 0 2
Hence, L11 cannot be proved in [L1-L10, MP]. Q.E.D.
The results file #03 proves also the following
Theorem 2.8.5 (thanks to Pavels Mihailovs for the correction). The following (classically provable)
formulas cannot be proved in the constructive propositional logic [L1-L10, MP]:
¬¬A → A
(¬B → ¬A) → (A→B)
(¬A→B)→(¬B→A)
(¬¬A → ¬¬B) → (A→B)
(A→B) → ¬AvB
((A→B)→B)→ AvB
((A→B)→A)→A
¬(A&¬B)→(A→B)
¬(A→B)→A&¬B
Av(A→B)
(A→B)→((¬A→¬B)→(B→A))
Indeed, all these formulas take non-"true" values under the "truth tables" from the proof of Theorem
2.8.4.
The following three formulas also cannot be proved in the constructive propositional logic, yet,
unfortunately, the "truth tables" from our proof of Theorem 2.8.4 do not allow proving this:
¬(A&B) → ¬A v ¬B
¬¬(AvB) → ¬¬A v ¬¬B
(A→B)v(B→A)
Indeed, under the above "truth tables", these formulas always take "true" values (see results file #03).
Another interesting conclusion: add these three formulas as additional axioms to [L1-L10, MP] – and
L11 will remain still unprovable!
Thus, we did not succeed in building a three-valued logic that would allow showing that the latter three
formulas cannot be proved in the constructive propositional logic. Is it possible at all to build a multi-
valued logic that would separate constructively provable propositional formulas from unprovable ones?
Kurt Gödel showed in 1932 that this is impossible: none of the finitely-valued logics "matches" exactly
the constructive propositional logic:
K. Gödel. Zum intuitionistischen Aussagenkalkül, Anzeiger Akademie der Wissenschaften Wien, Math.-naturwiss. Klasse,
1932, Vol. 69, pp.65-66.
Exercise 2.8.4. a) (For smart students) Verify that the latter three formulas cannot be proved in the
constructive propositional logic [L1-L10, MP]. Or, see Section 4.4.
b) Verify that any of the following formulas could be used - instead of Bv¬B – as the axiom L11 of the
classical propositional logic: i) (A→B) → ¬AvB, ii) ¬¬B→B, iii) ¬(A→B)→A (Hint: since all these
formulas are provable in [L1-L11, MP], it remains to prove L11 in [L1-L10, MP] + (i), in [L1-L10, MP] +
(ii), and in [L1-L10, MP] + (iii)).
c) Verify that with ¬¬B→B instead of L11 the "crazy" axiom L10 becomes 100% derivable from the other
axioms. Perhaps, this is why many textbooks prefer the combination L1-L9 + ¬¬B→B as the axiom list
for the classical propositional logic. But, then, we are forced to define the constructive propositional logic
not as a subset of the classical one, but as the classical logic with the axiom ¬¬B→B replaced by the
"crazy" axiom L10: ¬B→(B→C)!
Axiom L10 again...
Finally, let us check which of the main results of Sections 2.5 (constructive logic) and 2.6 (classical logic)
depend on the "crazy" axiom L10: ¬A→(A→B). Let your program print out only those variants of "truth
tables" that make "true" all the axioms L1-L8, and make: L9 - "true", L10 - not "true". My program yields
6 such variants, see the results file #04. Surprisingly, in all these variants L11 also is "true" (thus, the
results file #04 equals the results file #02). As the most productive appears
Implication variant #1:
2 2 2 0 2 2 0 1 2 L1-L8 true.
Variant #1. Negation: 2 2 1 L9 true. L10 not true. L11 true.
Constructively provable formulas
Not true: (AvB)→((¬A)→B)
Not true: ((¬A)vB)→(A→B)
Not true: ((¬¬A)→(¬¬B))→(¬¬(A→B))
Not true: (¬¬A)→((¬A)→A)
Not true: (Av(¬A))→((¬¬A)→A)
Not true: ¬¬((¬¬A)→A)
Classically provable formulas
True: (¬¬(AvB))→((¬¬A)v(¬¬B))
True: (¬(A&B))→((¬A)v(¬B))
Not true: (¬¬A)→A
Not true: ((¬B)→(¬A))→(A→B)
Not true: ((¬A)→B)→((¬B)→A)
Not true: ((¬¬A)→(¬¬B))→(A→B)
True: (A→B)→((¬A)vB)
Not true: ((A→B)→B)→(AvB)
Not true: ((A→B)→A)→A
Not true: (¬(A&(¬B)))→(A→B)
True: (A→B)→(((¬A)→B)→B)
Not true: (¬(A→B))→(A&(¬B))
Not true: Av(A→B)
True: (A→B)v(B→A)
Not true: (A→B)→(((¬A)→(¬B))→(B→A))
Thus, the following constructively provable formulas cannot be proved in the minimal logic [L1-L9, MP]
(and even in [L1-L9, L11, MP]), i.e. they cannot be proved without the "crazy" axiom L10:
(AvB)→(¬A→B)
¬AvB → (A→B)
(¬¬A→¬¬B) → ¬¬(A→B)
¬¬A → (¬A→A)
Av¬A → (¬¬A→A)
¬¬(¬¬A→A)
And the following classically provable formulas cannot be proved without the "crazy" axiom L10 (thanks
to Pavels Mihailovs for the correction):
¬¬A→A
(¬B→¬A)→(A→B)
(¬A→B)→(¬B→A)
(¬¬A→¬¬B)→(A→B)
((A→B)→B) → AvB
((A→B)→A)→A
¬(A&¬B)→(A→B)
¬(A→B)→A&¬B
Av(A→B)
(A→B)→((¬A→¬B)→(B→A))
Exercise 2.8.5 (thanks to Stanislav Golubcov for the idea). But how about the remaining four (classically
provable) formulas:
a) (A→B) → ¬AvB
b) ¬(A&B) → ¬Av¬B,
c) ¬¬(AvB) → ¬¬Av¬¬B
c1) (A→B)→((¬A→B)→B)
d) (A→B)v(B→A)?
Show that the formulas (a, b, c) can be proved without the "crazy" axiom L10, i.e prove them in [L1-L9,
L11, MP]. (Hint: use Theorem 2.6.4 (a) [L1-L8, MP]: Av¬A |- (A→B)→¬AvB.). For smart students: how
about the remaining formulas (c1, d)?
Using computers in mathematical proofs
Do you trust the above proofs? Personally, I trust much more my ability to write (relatively) error-free computer programs
than my ability to carry out error-free mathematical proofs. But how about you? Of course, you do not need trusting my (or
your own) program generating the results files #00, #01, #02, #03 and #04. We used these files only to select the "truth table"
variants allowing to prove our independence results. The critical points to be trusted are (see my implementation file) : a) the
recursive program
int MyFormula::ValueAt(int A, int B, int C)
and b) the character string analyzer
int MyFormula::Analyze(int *pOperation, AnsiString *pSubFormula1, AnsiString *pSubFormula2)
You may wish to remove your worries by verifying directly that under all the 3 truth table variants used above: a) the axioms
L1-L8 are true, and b) the axioms L9, L10, L11 and other formulas are true or not true according to the goal of each particular
proof. Before you have performed this 100%, you can feel the flavor of using computers in mathematical proofs (I trust this
proof, I do not trust it, I trust this proof, I do not trust it, I trust this proof, I do not trust it, I trust this proof, I do not trust it, I
trust this proof, I do not trust it, I trust this proof, I do not trust it, I trust this proof, I do not trust it, I trust this proof, I do not
trust it, I trust this proof, I do not trust it, I trust this proof, I do not trust it, I trust this proof, I do not trust it, I trust this proof, I
do not trust it,...)
Unfortunately, in more complicated cases the situation does not allow the above simple exit (i.e. manual verification of the
solution found by a computer):
"The Four Colour Theorem was the first major theorem to be proved using a computer, having a proof that could not be
verified directly by other mathematicians. Despite some worries about this initially, independent verification soon convinced
everyone that the Four Colour Theorem had finally been proved. Details of the proof appeared in two articles in 1977. Recent
work has led to improvements in the algorithm." (According to the article: The Four Colour Theorem in MacTutor History of
Mathematics archive).
The proof of the Four Colour Theorem was completed in 1976 by Kenneth Appel and Wolfgang Haken, see their photographs
published in European Mathematical Society, Newsletter No. 46, December 2002, pp. 15-19.
"The best-known, and most debated, instance is the use of computer analysis by Kenneth Appel and Wolfgang Haken of the
University of Illinois in their 1976 proof of the four-colour conjecture (that four colours suffice to colour in any map drawn
upon a plane in such a way that countries which share a border are given different colours). First put forward in 1852, the
conjecture had become perhaps the most famous unsolved problem in mathematics, resisting a multitude of efforts at proof for
over a century. Appel and Haken's demonstration rested upon computerized analysis, occupying 1,200 hours of computer time,
of over 1,400 graphs. The analysis of even one of those graphs typically went beyond what an unaided human being could
plausibly do: the ensemble of their demonstration certainly could not be checked in detail by human beings. In consequence,
whether that demonstration constituted "proof" was deeply controversial..." (according to
Donald MacKenzie. Computers and the Sociology of Mathematical Proof. In: Trends in the History and Philosophy of
Mathematics, Odense: University of Southern Denmark, 2004, pp.67-86).
See also
Ken Appel on the 4CT proof, December 1998
Robin Wilson. Four Colours Suffice. European Mathematical Society, Newsletter No. 46, December 2002, pp. 15-19 (online
copy).
The Four Color Theorem, November 13, 1995, by Robin Thomas.
A computer-checked proof of the Four Colour Theorem, 2004, by Georges Gonthier.
Doron Zeilberger. Opinion 54: It is Important to Keep Looking for Non-Computer Proofs of the Four-Color Theorem, But Not
For the "Usual" Reasons (available online at http://www.math.rutgers.edu/¬zeilberg/Opinion54.html).
Two other famous computer assisted mathematical proofs:
- In 1989, by using Cray super-computers, Clement W. H. Lam finished his proof that finite projective plane of order 10 is
impossible (for details see Projective plane in Wikipedia, the free encyclopedia).
- In 1998, Thomas C. Hales finished his proof of Kepler conjecture about the densest arrangement of spheres in space
(Johannes Kepler proposed it in 1611, for details see Kepler conjecture in Wikipedia, the free encyclopedia).
See logical software links selected by Peter Suber.
Visit the Mizar Project.
